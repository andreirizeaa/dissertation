\documentclass[12pt,a4paper]{report}

% ============================================
% PACKAGES
% ============================================

% Encoding and fonts
\usepackage[utf8]{inputenc}
\usepackage[T1]{fontenc}
\usepackage{lmodern}

% Page layout
\usepackage[margin=2.5cm]{geometry}
\usepackage{setspace}
\onehalfspacing

% Graphics and figures
\usepackage{graphicx}
\usepackage{float}
\usepackage{subcaption}
\graphicspath{{figures/}}

% Tables
\usepackage{booktabs}
\usepackage{longtable}
\usepackage{tabularx}
\usepackage{multirow}

% Math
\usepackage{amsmath}
\usepackage{amssymb}

% Code listings
\usepackage{listings}
\usepackage{xcolor}

\definecolor{codegreen}{rgb}{0,0.6,0}
\definecolor{codegray}{rgb}{0.5,0.5,0.5}
\definecolor{codepurple}{rgb}{0.58,0,0.82}
\definecolor{backcolour}{rgb}{0.95,0.95,0.92}

\lstdefinestyle{mystyle}{
    backgroundcolor=\color{backcolour},
    commentstyle=\color{codegreen},
    keywordstyle=\color{magenta},
    numberstyle=\tiny\color{codegray},
    stringstyle=\color{codepurple},
    basicstyle=\ttfamily\footnotesize,
    breakatwhitespace=false,
    breaklines=true,
    captionpos=b,
    keepspaces=true,
    numbers=left,
    numbersep=5pt,
    showspaces=false,
    showstringspaces=false,
    showtabs=false,
    tabsize=2
}
\lstset{style=mystyle}

% Hyperlinks
\usepackage[hidelinks]{hyperref}
\usepackage{url}

% Bibliography
\usepackage[
    backend=biber,
    style=authoryear,
    sorting=nyt,
    maxcitenames=2
]{biblatex}
\addbibresource{references.bib}

% Glossaries and acronyms
\usepackage[acronym,toc]{glossaries}
\makeglossaries

% Headers and footers
\usepackage{fancyhdr}
\pagestyle{fancy}
\fancyhf{}
\fancyhead[L]{\leftmark}
\fancyhead[R]{\thepage}
\renewcommand{\headrulewidth}{0.4pt}

% Appendices
\usepackage[toc,page]{appendix}

% ============================================
% ACRONYMS
% ============================================

\newacronym{pb}{PB}{Pitch Book}
\newacronym{ib}{IB}{Investment Banking}
\newacronym{rag}{RAG}{Retrieval-Augmented Generation}
\newacronym{llm}{LLM}{Large Language Model}
\newacronym{vlm}{VLM}{Vision Language Model}
\newacronym{pm}{PM}{Precedent Material}
\newacronym{api}{API}{Application Programming Interface}
\newacronym{sdlc}{SDLC}{Software Development Lifecycle}
\newacronym{poc}{POC}{Proof of Concept}
\newacronym{hitl}{HITL}{Human-in-the-Loop}

% ============================================
% DOCUMENT INFO
% ============================================

\title{Compliance Aware, Agentic Pitch Book Generation For Bankers}
\author{Andrei Rizea}
\date{\today}

% ============================================
% DOCUMENT
% ============================================

\begin{document}

% --------------------------------------------
% FRONT MATTER
% --------------------------------------------

% Title Page
\begin{titlepage}
    \centering
    \vspace*{2cm}

    {\Huge\bfseries Compliance Aware, Agentic Pitch Book Generation For Bankers\par}

    \vspace{1cm}

    {\Large An all-in-one content generation tool providing agentic workflows with RAG search capabilities.\par}

    \vfill

    {\Large A Queen Mary Final Year Dissertation\par}

    \vspace{1cm}

    {\large\bfseries Andrei Rizea\par}
    {\large Student ID: 220300153\par}

    \vspace{1cm}

    {\large Supervised by: Manesha Peiris\par}

    \vspace{1cm}

    {\large Programme of Study:\par}
    {\large Digital \& Technology Solutions (Software Engineering)\par}

    \vspace{1cm}

    {\large Module: IOT635W\par}

    \vfill

    \vspace{1cm}

    {\large Queen Mary University of London\par}
    {\large \today\par}

\end{titlepage}

% Abstract
\begin{abstract}
\addcontentsline{toc}{chapter}{Abstract}
% TODO: Write abstract
\end{abstract}

% Acknowledgements
\chapter*{Acknowledgements}
\addcontentsline{toc}{chapter}{Acknowledgements}
% TODO: Write acknowledgements

% Table of Contents
\tableofcontents
\listoffigures
\listoftables
\printglossary[type=\acronymtype,title=List of Acronyms]

% --------------------------------------------
% MAIN MATTER
% --------------------------------------------

\chapter{Introduction, Scope and Context}
\label{ch:introduction}

Knowledge workers in contemporary organisations spend a disproportionate fraction of their time not generating knowledge but searching for, reformatting, and repackaging existing knowledge assets \parencite{markus2001knowledge}. This observation, drawn from research into knowledge reuse in organisational settings, resonates with particular force in investment banking, where the production of client-facing presentation materials---commonly termed pitch books---consumes substantial analyst time despite the highly repetitive nature of these documents. This chapter establishes the problem context, situates the proposed solution within relevant literature, and defines the scope of this proof-of-concept implementation.

\section{The Cognitive Economics of Investment Banking}
\label{sec:cognitive-economics}

Investment banking represents a paradigmatic knowledge industry, with professional value derived primarily from analytical insight and client relationship management rather than physical output. Yet empirical evidence suggests a misallocation of cognitive resources within this sector. Industry surveys indicate that junior analysts dedicate between 50 and 70 percent of their working hours to document preparation activities, with pitch book creation representing a substantial component of this workload \parencite{cfi2025analyst, wso2024hours}. This allocation persists despite pitch books exhibiting high structural similarity---variations typically involve updated financial metrics, adjusted company names, and modified date ranges rather than fundamentally novel analytical frameworks.

The work of investment banking analysts divides into categories of markedly different cognitive demand. High-value activities include financial modelling, valuation analysis, due diligence investigation, and client advisory work. Lower-value but time-consuming activities encompass data gathering from disparate sources, formatting content to institutional templates, and iterative revision cycles addressing stylistic rather than substantive concerns. The current distribution of effort inverts what might be considered optimal from either a productivity or professional development perspective.

Radhakrishna and colleagues, investigating knowledge management processes in investment banking through structural equation modelling, found that organisations implementing robust knowledge management systems achieved productivity improvements of 20 to 25 percent \parencite{radhakrishna2024km}. Their research specifically identified document production workflows as presenting significant optimisation opportunities, particularly where repetitive processes could be augmented through intelligent automation while preserving the expert judgement essential to client-facing deliverables.

\section{Problem Analysis: Knowledge Reuse Barriers in Document Production}
\label{sec:problem-analysis}

The inefficiencies observable in pitch book production reflect broader challenges in organisational knowledge management that Markus characterised as knowledge reuse problems \parencite{markus2001knowledge}. Markus distinguished between different knowledge reuse situations---including shared work producers, shared work practitioners, expertise-seeking novices, and secondary knowledge miners---each presenting distinct barriers to effective reuse. Pitch book production involves elements of multiple categories: analysts reuse their own prior work, seek to leverage colleagues' precedent materials, and attempt to capture institutional knowledge embedded in historical documents.

Three interconnected barriers impede the realisation of this reuse potential. The first concerns the misallocation of cognitive resources described above. When analysts spend hours locating relevant precedent materials, extracting applicable sections, and reformatting content to current requirements, they engage in retrieval and transformation tasks that add minimal intellectual value. This time could alternatively support deeper analytical work or expanded client engagement.

The second barrier involves the erosion of institutional memory through inadequate knowledge codification. Knowledge management theory distinguishes between tacit knowledge---experiential understanding that resists explicit articulation---and explicit knowledge that can be documented and transmitted \parencite{dalkir2005km}. Pitch books represent attempts to codify analytical insights into explicit form, yet the institutional knowledge about effective pitch book construction---including template conventions, narrative structures, and persuasive framings---often remains tacit, residing in individual practitioners' experience rather than accessible repositories.

The third barrier relates to the friction inherent in transferring tacit knowledge to new practitioners. Each institution maintains distinctive approaches to pitch book construction, encompassing template designs, data sourcing procedures, narrative conventions, and quality expectations. Junior analysts must acquire this tacit knowledge through observation and iterative feedback, a process that extends onboarding periods and generates revision cycles that consume senior banker time. The absence of systematic knowledge capture mechanisms means this learning process repeats with each new cohort.

\section{Gap Analysis: The Evolution of Presentation Automation}
\label{sec:gap-analysis}

Research in automated presentation generation has progressed through distinct paradigms, each addressing different dimensions of the challenge. Early approaches, emerging around 2014, focused primarily on text-to-slide conversion---transforming written content into slide format through layout algorithms and content segmentation \parencite{zheng2025pptagent}. These systems prioritised content organisation over visual design, producing functional but aesthetically limited outputs. Subsequent developments introduced template-based generation, where predefined slide structures constrained output formatting, improving visual consistency at the cost of flexibility.

The recent PPTAgent framework advances beyond this paradigm by adopting an edit-based approach that analyses reference presentations to extract structural patterns, then applies these patterns to new content \parencite{zheng2025pptagent}. This represents a significant conceptual shift: rather than generating slides from abstract principles, the system learns from concrete examples. PPTAgent demonstrates superior performance across content quality, design coherence, and narrative structure compared to text-to-slide alternatives. The framework's evaluation methodology, employing both automated metrics and human assessment, establishes benchmarks applicable to domain-specific implementations.

However, PPTAgent and comparable systems target general presentation scenarios, leaving domain-specific requirements unaddressed. Investment banking pitch books involve specialised content types---financial tables, transaction comparables, market positioning charts---that generic systems handle inadequately. Furthermore, institutional template compliance represents a hard constraint in professional contexts that academic research has not prioritised. A pitch book that violates corporate visual identity standards fails regardless of content quality.

Commercial platforms offering AI-assisted presentation creation similarly address generic business contexts. These tools provide value for general corporate presentations but lack the domain knowledge, data integration capabilities, and template fidelity required for investment banking applications. The gap between available solutions and professional requirements creates the opportunity this project addresses.

Retrieval-Augmented Generation provides an architectural foundation for grounding generative outputs in authoritative knowledge sources \parencite{lewis2020rag}. By coupling large language models with retrieval mechanisms that surface relevant documents, RAG architectures reduce hallucination risks and improve factual accuracy. For pitch book generation, RAG offers potential for incorporating precedent materials, ensuring generated content aligns with institutional conventions and historical approaches. However, implementing effective retrieval requires substantial infrastructure for document processing, embedding generation, and similarity search---investments that may exceed proof-of-concept scope.

Agentic AI architectures extend these capabilities by enabling autonomous planning and execution of multi-step processes. Rather than responding to single prompts, agentic systems decompose complex tasks, invoke appropriate tools, and synthesise results \parencite{ibm2024agentic}. For pitch book generation, agentic approaches enable workflows that gather financial data from market sources, retrieve relevant precedent materials, generate content specifications, and assemble formatted outputs---all orchestrated without continuous human intervention. This aligns with the broader industry trajectory toward agentic enterprise systems \parencite{mckinsey2025agentic}.

The synthesis of these capabilities---template-aware generation, domain-specific data retrieval, and agentic orchestration---defines the technical approach this project explores. The novel contribution lies not in advancing any individual capability but in demonstrating their integration for a specific professional domain with stringent compliance requirements.

\section{Aims and Objectives}
\label{sec:aims-objectives}

This project aims to design, implement, and evaluate a proof-of-concept system demonstrating how AI-powered document generation can address knowledge reuse barriers in investment banking pitch book production while maintaining institutional template compliance.

From an organisational perspective, the aim is to demonstrate a workflow enabling analysts to generate substantive first drafts of pitch books through minimal input specification, reducing time allocation to document formatting while preserving human oversight of analytical content. This supports the broader objective of reallocating cognitive resources toward higher-value activities.

From an educational perspective, the aim is to investigate the technical integration of large language model orchestration with programmatic document generation, contributing understanding of how AI capabilities can be applied to professional document production contexts with specific compliance constraints.

These aims translate into five concrete objectives:

\begin{enumerate}
    \item Design and implement a template analysis component that programmatically extracts layout structures, colour palettes, font specifications, and placeholder configurations from reference PowerPoint files, enabling generated content to maintain visual consistency with institutional standards.

    \item Develop an agentic data retrieval layer that autonomously gathers company financials, market data, and regulatory filings from public sources including Yahoo Finance and SEC EDGAR, reducing manual data collection effort.

    \item Construct an LLM-powered content planning module that generates structured specifications for slide content, adapting output to different pitch book types while maintaining narrative coherence appropriate to investment banking conventions.

    \item Build a slide assembly component that maps planned content to extracted template structures, producing PowerPoint files that comply with institutional formatting requirements.

    \item Evaluate system outputs against metrics encompassing generation efficiency, template compliance, and content appropriateness, establishing baseline performance for future development.
\end{enumerate}

\section{Scope, Constraints and Success Criteria}
\label{sec:scope}

The project scope encompasses a functional proof of concept demonstrating the complete workflow from user input specification through formatted pitch book output. The system accepts deal context parameters (target company, transaction type, relevant date ranges), a reference template file, and pitch book type specification, producing a PowerPoint file populated with retrieved data and generated content structured according to the template design.

Several constraints define implementation boundaries. The agentic orchestration layer employs established frameworks rather than custom agent architectures, prioritising integration demonstration over novel agent development. Template analysis relies on programmatic extraction through python-pptx rather than vision-language model interpretation, deferring visual understanding capabilities to future work. Data sources are limited to publicly accessible APIs, excluding proprietary market data terminals that would be available in production contexts. Retrieval-Augmented Generation against precedent material repositories is designated as a stretch goal rather than core requirement, acknowledging the infrastructure investment this capability demands.

The project explicitly excludes real-time market data integration, production deployment infrastructure, and comprehensive regulatory compliance validation. These represent essential considerations for production systems but exceed proof-of-concept scope. Similarly, the system does not attempt to replace analyst judgement on deal positioning, valuation conclusions, or client-specific customisation---these remain firmly within human responsibility.

Success evaluation applies four criteria. First, the system should generate complete pitch book drafts within five minutes of input specification, demonstrating practical utility for workflow integration. Second, generated outputs should maintain template fidelity, with formatting, colours, and layouts matching reference files within assessable tolerances. Third, retrieved financial data should accurately reflect source information, with verification against manual retrieval establishing correctness baselines. Fourth, content structure should align with investment banking conventions, assessed through comparison with precedent examples and, where available, practitioner feedback.

Throughout, the system is designed for human-in-the-loop operation, producing drafts that support rather than supplant professional judgement. The objective is augmentation---enabling analysts to begin from substantive starting points rather than blank slides---not automation of the complete pitch book production process.

% --------------------------------------------
% SUBSEQUENT CHAPTERS (placeholders)
% --------------------------------------------

\chapter{Literature Review}
\label{ch:literature-review}

% TODO: Expand literature review

\chapter{Methodology}
\label{ch:methodology}

% TODO: Methodology content

\chapter{System Design}
\label{ch:system-design}

% TODO: System design content

\chapter{Implementation}
\label{ch:implementation}

% TODO: Implementation details

\chapter{Testing and Evaluation}
\label{ch:testing}

% TODO: Testing methodology and results

\chapter{Results and Discussion}
\label{ch:results}

% TODO: Present and discuss results

\chapter{Conclusion}
\label{ch:conclusion}

% TODO: Conclusions and future work

% --------------------------------------------
% BACK MATTER
% --------------------------------------------

% Bibliography
\printbibliography[heading=bibintoc]

% Appendices
\begin{appendices}

\chapter{Risk Assessment}
\label{app:risk-assessment}

% TODO: Include risk assessment table

\chapter{Project Plan}
\label{app:project-plan}

% TODO: Include Gantt chart or project timeline

\chapter{KSB Mapping}
\label{app:ksb-mapping}

% TODO: Map project to apprenticeship Knowledge, Skills, and Behaviours

\end{appendices}

\end{document}
